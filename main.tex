%%%%%%%%%%%%%%%%%%%%%%%%%%%%%%%%%%%%%%%%%
% Journal Article
% LaTeX Template
% Version 1.4 (15/5/16)
%
% This template has been downloaded from:
% http://www.LaTeXTemplates.com
%
% Original author:
% Frits Wenneker (http://www.howtotex.com) with extensive modifications by
% Vel (vel@LaTeXTemplates.com)
%
% License:
% CC BY-NC-SA 3.0 (http://creativecommons.org/licenses/by-nc-sa/3.0/)
%
%%%%%%%%%%%%%%%%%%%%%%%%%%%%%%%%%%%%%%%%%

%----------------------------------------------------------------------------------------
%	PACKAGES AND OTHER DOCUMENT CONFIGURATIONS
%----------------------------------------------------------------------------------------

\documentclass[twoside,twocolumn]{article}

\usepackage{blindtext} % Package to generate dummy text throughout this template

\usepackage[sc]{mathpazo} % Use the Palatino font
\usepackage[T1]{fontenc} % Use 8-bit encoding that has 256 glyphs
\linespread{1.05} % Line spacing - Palatino needs more space between lines
\usepackage{microtype} % Slightly tweak font spacing for aesthetics

\usepackage[english]{babel} % Language hyphenation and typographical rules

\usepackage[hmarginratio=1:1,top=32mm,columnsep=20pt]{geometry} % Document margins
\usepackage[hang, small,labelfont=bf,up,textfont=it,up]{caption} % Custom captions under/above floats in tables or figures
\usepackage{booktabs} % Horizontal rules in tables

\usepackage{lettrine} % The lettrine is the first enlarged letter at the beginning of the text

\usepackage{enumitem} % Customized lists
\setlist[itemize]{noitemsep} % Make itemize lists more compact

\usepackage{abstract} % Allows abstract customization
\renewcommand{\abstractnamefont}{\normalfont\bfseries} % Set the "Abstract" text to bold
\renewcommand{\abstracttextfont}{\normalfont\small\itshape} % Set the abstract itself to small italic text

\usepackage{titlesec} % Allows customization of titles
\renewcommand\thesection{\Roman{section}} % Roman numerals for the sections
\renewcommand\thesubsection{\roman{subsection}} % roman numerals for subsections
\titleformat{\section}[block]{\large\scshape\centering}{\thesection.}{1em}{} % Change the look of the section titles
\titleformat{\subsection}[block]{\large}{\thesubsection.}{1em}{} % Change the look of the section titles

\usepackage{datetime}
\newdateformat{monthyear}{\monthname[\THEMONTH] \THEYEAR}

\usepackage{fancyhdr} % Headers and footers
\pagestyle{fancy} % All pages have headers and footers
\fancyhead{} % Blank out the default header
\fancyfoot{} % Blank out the default footer
\fancyhead[C]{Dialectic Engineering: an invitation to think deeply $\bullet$ \monthyear\today  } % Custom header text
\fancyfoot[RO,LE]{\thepage} % Custom footer text

\usepackage{titling} % Customizing the title section

\usepackage{hyperref} % For hyperlinks in the PDF

%----------------------------------------------------------------------------------------
%	TITLE SECTION
%----------------------------------------------------------------------------------------

\setlength{\droptitle}{-4\baselineskip} % Move the title up

\pretitle{\begin{center}\Huge\bfseries} % Article title formatting
\posttitle{\end{center}} % Article title closing formatting
\title{Dialectic Engineering: an invitation to think deeply} % Article title
\author{%
\textsc{James Cauwelier} \\[1ex] % Your name
%\normalsize Your fancy institution \\ % Your institution
\normalsize \href{mailto:james@accelerated.dev}{james@accelerated.dev} % Your email address
%\and % Uncomment if 2 authors are required, duplicate these 4 lines if more
%\textsc{Jane Smith}\thanks{Corresponding author} \\[1ex] % Second author's name
%\normalsize University of Utah \\ % Second author's institution
%\normalsize \href{mailto:jane@smith.com}{jane@smith.com} % Second author's email address
}
\date{\today} % Leave empty to omit a date
\renewcommand{\maketitlehookd}{%
  \begin{abstract}
    \noindent The dialectical method, best known as the Socratic dialogue, has a rich history from the old Greeks (Aristoteles and Plato), to modern times.
    A modern version of it enables us to investigate not only dialogues with others, but also internal ones with
    ourselves.
    Whenever thinking is involved, dialectics can be used to add depth and sophistication.
    Dialectic engineering refers to a new practice where dialectics is applied to the process of software engineering
    to critically reflect on what will be or has been built.
    Using modern frameworks, our thinking can be made into an object of analysis, exposing structures called thought
    -forms.
    This objectification provides a seeding ground for adding absent thought structures, which enriches thought with
    previously absent perspectives.
    Using this approach as a theory of mind, minds can be stretched into a state of higher fluidity, activate
    higher forms of systems thinking, and enable higher functioning by extending our time horizon in daily work.
    A higher time horizon enables a harmonization of long-term strategic interests with short-term productivity
    concerns, and a way to manage tensions balancing short-term with long-term stakeholder needs.
\end{abstract}
}

\usepackage{cite}
\usepackage[numbers]{natbib}
\usepackage{dialogue}

\setlength{\parskip}{1em}

%----------------------------------------------------------------------------------------

\begin{document}

% Print the title
\maketitle

%----------------------------------------------------------------------------------------
%	ARTICLE CONTENTS
%----------------------------------------------------------------------------------------

\section{Introduction}

The dialectical method refers originally to dialogue between people holding different points of view about a subject
but wishing to arrive at the truth through reasoned argument~\citep{wikipedia_dialectic}.
Hegel provides a newer, expanded interpretation by also considering disagreements a person can have on their self
-constructed views~\citep{wikipedia_dialectic}.
In other words, a dialogue can happen where one disagrees with their own views thereby updating their current view
with new insights.
While producing software, engineers engage in such a dialectic where written software is reviewed by both themselves
or peers in a code review process.
Unfortunately, not many pause to think how they can be maximally constructively critical of their own thoughts
and, by extension, the software they produce.

The approach introduced in this brief allows an individual to critique their thinking in a methodical way and
helps to uncover blind spots.
Ideally, a person enlists the help of others to build critical thinking skills.
This brief invites you to join limited groups of 2 - 4 individuals engaging in sessions of dialectic software
engineering.
To join, send an email to \href{mailto:james@accelerated.dev}{james@accelerated.dev} or reach out on \href{https://www.linkedin.com/in/jamescauwelier/}{LinkedIn}.

A dialectic software session starts with selecting a topic of conversation.
Next, using the roles of mob programming sessions, a navigator will express knowledge of the topic and instruct the
driver on how to implement a code representation of it.
After a brief period of implementation, dialectic reflection will match the expressed knowledge to thought-forms,
abstract description of the structure of a particular thought.
There might be different competing thought forms in play, where we'll try to identify the most prominent thought form
that is currently being expressed.
From this, a process of mind opening or mind stretching is started.
Knowing the presence of thought-forms, and in particular the absence of others, provides the opportunity to introduce
new ones by the formulation of mind-opening questions.
An example of such questions will be presented later.

\section{Methods of Deliberate Development}

First, let's talk about the why of \emph{Dialectic Engineering}, which is appropriately described as a deliberate
developmental practice as introduced by Kegan and Lahey~\citep{kegan2016everyone}.
Often, developmental practices are mistaken for learning activities, so let's start by creating this distinction.

Laske refers to two types of developmental: vertical and horizontal~\citep{laske_MHD_vol_II}.
The first one, horizontal development, is behavioral.
This is where we learn new skills and is what most mean when discussing professional development: learning a new programming language (e.g. Go,
Typescript, Rust, etc.), a framework (e.g. Spring boot, React), a methodology (e.g. Domain Driven Design, Test Driven
Development), a paradigm (e.g. functional programming).

But true development is vertical and implies new levels of emotional and cognitive development are reached.
Learning happens when what we know changes.
True vertical development happens when how we know changes.
Using a Hegelian dialectic conversation, a topic can be detailed across 4 categories of thought-forms where the
introduction and deepening of a thought-form is an act of vertical cognitive development.
Participants need the emotional maturity to present with humility and a readiness to receive critical inquiry, after
which the rewards will be plenty.

\subsection{Thought Forms}

The thought-forms we use in our developmental model are separated according to the 4 moments of the dialectic~\citep{
    laske_2023_4_moments} where each focuses on a different category of thought:

\begin{itemize}
    \item \emph{Process}: focus on identity, change, history and evolution of thought, negation
    \item \emph{Context}: focus on parts, collections, and organizations thereof in stable constellations
    \item \emph{Relation}: focus on relating items using common ground, detailing relationships, and eventually
    seeing relationship as constitutive (being foundational of what it relates)
    \item \emph{Transformation}: focus on instabilities as the driver of transformation, systems of systems, open
    incompletely defined systems, etc.
\end{itemize}

Most focus their thoughts almost exclusively on contextual thinking.
And all of these thought-forms are abstract and sometimes elusive, but that's only because we are not accustomed to
take our own thinking as a subject of thought.
We're looking to change that to guide us to some real tangible and practical benefits.

In our sessions, we'll introduce the categories and thought-forms one by one, and practice them together.
To do this, we need another tool, the thinking spaces.

\subsection{Thinking Spaces}\label{subsec:Thinking Spaces}

A thinking space is an artificial delineation of where to apply our focus.
The distinction is artificial because our thinking doesn't follow such clear boundaries.
Focussing on a few defined thinking spaces allows us to direct attention, and protect us from engaging too much with
unproductive, philosophical outputs.

As an engineer, I get exposed to three thinking spaces:

\begin{itemize}
    \item \emph{Software (or solution domain)}: I think about how to solve challenges by writing software systems
    \item \emph{Problem Domain}: I think about the customer or users' needs and understand them deeply
    \item \emph{Organization}: I think about how the team can organize itself around problem and solution, and how
    this organization helps our hurts our progress.
\end{itemize}

It helps to know in what space we're thinking.
To practice the thought forms, we start in the solution space and make assumptions on the other spaces.
In applied practice, however, dialectic engineering will need to focus on all three dimensions and on different
levels of abstraction.
\section{Mind Stretching: a brief example}

This section illustrates a potential example of mind-opening dialogue.
The case is real, but the dialogue is shortened for practical reasons.

\subsection{Drafting legal documents}

The client is a lawyer wishing to automate the drafting of legal documents, such as contracts.
A rule-engine is built, which can decide to draft a piece of content or decide to query for more information.
The rule-engine can be edited on a CMS where a rule consists of a matching condition, and an action.
Rules are ordered by importance and the order of appearance in the document.

\subsection{An inflexible rule engine}

The challenge presented by our client is that the rule engine is inflexible and full of bugs.
When the lawyer notices a problem with a document, and reviews the rule engine, they often find it challenging to
confirm whether the rule engine has a bug or whether the rules are incorrect.
The complexity of putting all these rules together quickly got out-of-hand, leading to a pile of bug reports that end
up being misconfigurations the engineer is unable to charge for.
The end result is that the lawyer needs the engineer to listen to the requirement in order to verify the rule engine
and make adaptations.
This defeats the purpose of the software, which was to put control in the hands of the lawyer to direct a drafting
engine that saves them time and money.

\subsection{A mind stretching dialogue}

Here's the mind-opening dialogue between our client lawyer and software engineer.
A 3rd party, the facilitator, is asking mind-opening questions.

\begin{dialogue}
    \speak{Lawyer} We're seeing the same problem again. According to rules x and y, this paragraph shouldn't have
    been added. Additionally, further down the document, a critical piece is missing.
    \speak{Engineer} Yeah, but if you consider these rules, the engine is behaving as expected.
    \speak{Lawyer} The case doesn't apply here, because my client is a senior citizen and over 65 years old.
    \speak{Engineer} The system doesn't have that data. And so obviously, it can't act on it.
    \speak{Lawyer} Then, question should've been available in the system.
    \speak{Engineer} I can add it now, but I couldn't have known to add it unless you briefed me on it.
    \speak{Lawyer} But it's impossible for us to know all the questions we need to ask.
    \speak{Engineer} But why is that? You just told me the question that was missing here.
    \speak{Lawyer} Yes, but I only knew it was missing when being presented with an error in the system's outputs.
    There's a lot of similar questions and conditions under which we should ask them. The law also changes and so the
    system needs to change with it.
    \speak{FACILITATOR} It sounds like the current system is unstable because the outputs are incorrect?
    \speak{Lawyer} Yes
    \speak{FACILITATOR} And these instabilities are either a newly exposed incompleteness or a change in legal
    requirements?
    \speak{Lawyer} Yes
    \speak{Engineer} And they interrupt my tasks preventing me to do any real feature development.
    \speak{FACILITATOR} I see. And we don't have a way of knowing which instabilities will surface in the system
    tomorrow, because we don't know how the law will change or what unique use cases will present them to your lawyer
    practice?
    \speak{Lawyer} Unfortunately correct.
    \speak{FACILITATOR} I see. But one stability is that we know they will happen, right? We know there will be
    further incompleteness and law change?
    \speak{Lawyer} I fear so, yes.
    \speak{FACILITATOR} It appears to me as though these instabilities are an active driver through which the
    software evolves?
    \speak{Lawyer} Yes. Unfortunately, the software requirements can't be fully defined and I fear the automation
    attempt is doomed.
    \speak{FACILITATOR} How would you describe your original automation attempt or set of requirements?
    \speak{Lawyer} To automate the drafting of legal documents according to a set of rules.
    \speak{FACILITATOR} And what was really needed was to automate the drafting of legal documents according to a set
    of known and unknown rules? So the need to cater to unknown rules was not initially understood.
    \speak{Lawyer} Yeah, I guess so. But how would you do that?
    \speak{FACILITATOR} How are you doing it today?
    \speak{Laywer} We flag an issue to the engineer.
    \speak{Engineer} And then I need to dig into the rule engine, add questions, and different rules to reconfigure
    the system, after which we run the document again to test it.
    \speak{FACILITATOR} It appears to me that failure needs to be an inherent part of the system design where
    flagging them automatically reconfigures the system. And they aren't huge issues, but rather sources of truth the
    system relies on to correct itself.
    \speak{Engineer} Sure, but they take a lot of time to fix, and I don't have the bandwidth.
    \speak{FACILITATOR} So, let's dig into that and see if we can automate\ldots
\end{dialogue}

\subsection{A transformational solution}

What happened in this dialogue?
Early on the topic of instability surfaced.
A system can be in a stable state or an unstable one, and they often constantly move between the two.
This unstable state was seen as an insurmountable problem at first.
Both the engineer and the lawyer did not realize they had already produced a prototype of their future solution.
They had also not accepted the instability as an inherent part of their solution domain.
We don't tend to think of errors as solutions, but they're really two sides of the same coin.

The facilitator takes a more neutral stands towards errors and presents that the problem has already been solved.
The core of the proposition is to embed the errors as part of the system design.
Eventually the team would go on to produce a more elaborate document editor that tracks manual corrections and can
communicate with the lawyer where more information is needed.
For this purpose, the collected data is mapped onto actions by using a machine learning technique (decision trees).
The decision tree can tell when a valid state cannot be unambiguously be generated, which is when it can present
examples of situations to disambiguate.
The lawyer can then enter the question that needs to query for useful data to determine how to draft the document.
Each time an error is flagged, the decision tree is retrained.
When the law changes, older documents can be removed from the dataset and are no longer trained with.
A decision tree can also be inspected to verify the logic applied to arrive at a decision, allowing the lawyer to
correct system mistakes manually without intervention from the engineer.

Thought-forms are developmental.
Both the engineer were not able to access the needed thought-form, TF-23, 'value of conflict leading in a
developmental direction'.
This thought-form looks at conflict as a beneficial factor, rather than a disturbance.
Dialectical engineering is the practice where we gradually acquaint ourselves with all 28 thought forms as defined by
Laske~\citep{laske_2023_4_moments}, or a simplified set of 12 described by Basseches~\citep{shannon_metathinking}.
This then develops a catalogue of perspectives that enables us to escape into when we are stuck in our own thinking.
Much like a painter steps away from a piece to get a fresh look, we need to step away from our thinking to gain
perspective.
\section{Benefits}

The previous dialogue demonstrates how a facilitator can introduce thought-forms that are not yet accessible to
conversational partners.
Does this mean they have now developed to use this new cognitive skill?
Unfortunately not.
To develop cognitively, repeated sessions of mind-opening dialogue is needed to make new thought-forms available.
This reflective practice is best done in small groups that are developmentally diverse to maximally expose
participants to a wider variety of cognitive patterns.
But what is the benefit of acquiring these thought-forms?

The thinking spaces that were mentioned in \ref{subsec:Thinking Spaces} can be imagined as volumes that represent your
current thinking
space.
The more thought-forms your mind can wield, the larger the thinking space is.
The larger the solution and problem spaces, the larger their overlap, and the more potential for meeting customer
expectation.

Another way to illustrate benefit is to put it in terms of Elliot Jaques description of Requisite Organizations~\citep{jaques1989requisite}.
Jaques prescribes that an organization has different needs for individuals depending on the complexity requirements
of their work.
He then continues to say that individuals need to meet the capability requirements of their function, which can
concretely be described in terms of time horizon, the maximum amount of time a person needs to foresee in their
function.
A junior engineer functions mostly on a timescale of days or weeks as they worry only about resolving a specific
feature request or bug report.
The more senior the engineer, the more a strategic dimension is added to their work.
A principal engineer will take a long-term view on the software and, as an example, consider evolvability to protect
the long-term interests of the organization.
A time horizon here might span from months for prototypes and POC's to sometimes decades for core functionality in
mainframes of banking systems.

And so dialectical engineering is an interactive process to enable the expansion of mind-space to match time-horizon
requirements of software projects.
It provides the soft, missing piece on how to cognitively develop strategical engineering minds that can bridge the
technological
, real-world, and organizational dimensions of human work.

As an additional benefit, when we grow our own mind, we grow our immediate environment by association.
For example, accessing more thought-forms results in deeper discussions with customers where an engineer can
compensate for the absence of information by following the breadcrumbs in a customers' thought-structures.
\section{Conclusion}

Historically, engineers have developed themselves on their jobs in a horizontal direction using skill acquisition.
Joining a dialectic engineering cohort, engineers can develop vertically and unlock cognitive tooling to enter into
more strategic roles with longer time horizons.
This enlarges the range of responsibilities an engineer can take on from simple feature development, to entering into
deep customer discussions, and considering implications decades into the future.

In the past, these skills have been intuitively acquired and passed on through change encounters with cognitively
higher functioning individuals.
Using a dialectic engineering approach, once formalized, it can be more fairly shared with a larger audience.

Please reach out to join in this effort.

%----------------------------------------------------------------------------------------
%	REFERENCE LIST
%----------------------------------------------------------------------------------------

\bibliographystyle{plainnat} % Choose a bibliography style
\bibliography{references} % Include the BibTeX file

%----------------------------------------------------------------------------------------

\end{document}
