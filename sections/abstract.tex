\begin{abstract}
    \noindent The dialectical method, best known as the Socratic dialogue, has a rich history from the old Greeks (Aristoteles and Plato), to modern times.
    A modern version of it enables us to investigate not only dialogues with others, but also internal ones with
    ourselves.
    Whenever thinking is involved, dialectics can be used to add depth and sophistication.
    Dialectic engineering refers to a new practice where dialectics is applied to the process of software engineering
    to critically reflect on what will be or has been built.
    Using modern frameworks, our thinking can be made into an object of analysis, exposing structures called thought
    -forms.
    This objectification provides a seeding ground for adding absent thought structures, which enriches thought with
    previously absent perspectives.
    Using this approach as a theory of mind, minds can be stretched into a state of higher fluidity, activate
    higher forms of systems thinking, and enable higher functioning by extending our time horizon in daily work.
    A higher time horizon enables a harmonization of long-term strategic interests with short-term productivity
    concerns, and a way to manage tensions balancing short-term with long-term stakeholder needs.
\end{abstract}