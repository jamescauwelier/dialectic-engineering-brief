\section{Benefits}

The previous dialogue demonstrates how a facilitator can introduce thought-forms that are not yet accessible to
conversational partners.
Does this mean they have now developed to use this new cognitive skill?
Unfortunately not.
To develop cognitively, repeated sessions of mind-opening dialogue is needed to make new thought-forms available.
This reflective practice is best done in small groups that are developmentally diverse to maximally expose
participants to a wider variety of cognitive patterns.
But what is the benefit of acquiring these thought-forms?

The thinking spaces that were mentioned in \ref{subsec:Thinking Spaces} can be imagined as volumes that represent your
current thinking
space.
The more thought-forms your mind can wield, the larger the thinking space is.
The larger the solution and problem spaces, the larger their overlap, and the more potential for meeting customer
expectation.

Another way to illustrate benefit is to put it in terms of Elliot Jaques description of Requisite Organizations~\citep{jaques1989requisite}.
Jaques prescribes that an organization has different needs for individuals depending on the complexity requirements
of their work.
He then continues to say that individuals need to meet the capability requirements of their function, which can
concretely be described in terms of time horizon, the maximum amount of time a person needs to foresee in their
function.
A junior engineer functions mostly on a timescale of days or weeks as they worry only about resolving a specific
feature request or bug report.
The more senior the engineer, the more a strategic dimension is added to their work.
A principal engineer will take a long-term view on the software and, as an example, consider evolvability to protect
the long-term interests of the organization.
A time horizon here might span from months for prototypes and POC's to sometimes decades for core functionality in
mainframes of banking systems.

And so dialectical engineering is an interactive process to enable the expansion of mind-space to match time-horizon
requirements of software projects.
It provides the soft, missing piece on how to cognitively develop strategical engineering minds that can bridge the
technological
, real-world, and organizational dimensions of human work.

As an additional benefit, when we grow our own mind, we grow our immediate environment by association.
For example, accessing more thought-forms results in deeper discussions with customers where an engineer can
compensate for the absence of information by following the breadcrumbs in a customers' thought-structures.