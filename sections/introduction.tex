\section{Introduction}

The dialectical method refers originally to dialogue between people holding different points of view about a subject
but wishing to arrive at the truth through reasoned argument~\citep{wikipedia_dialectic}.
Hegel provides a newer, expanded interpretation by also considering disagreements a person can have on their self
-constructed views~\citep{wikipedia_dialectic}.
In other words, a dialogue can happen where one disagrees with their own views thereby updating their current view
with new insights.
While producing software, engineers engage in such a dialectic where written software is reviewed by both themselves
or peers in a code review process.
Unfortunately, not many pause to think how they can be maximally constructively critical of their own thoughts
and, by extension, the software they produce.

The approach introduced in this brief allows an individual to critique their thinking in a methodical way and
helps to uncover blind spots.
Ideally, a person enlists the help of others to build critical thinking skills.
This brief invites you to join limited groups of 2 - 4 individuals engaging in sessions of dialectic software
engineering.
To join, send an email to \href{mailto:james@accelerated.dev}{james@accelerated.dev} or reach out on \href{https://www.linkedin.com/in/jamescauwelier/}{LinkedIn}.

A dialectic software session starts with selecting a topic of conversation.
Next, using the roles of mob programming sessions, a navigator will express knowledge of the topic and instruct the
driver on how to implement a code representation of it.
After a brief period of implementation, dialectic reflection will match the expressed knowledge to thought-forms,
abstract description of the structure of a particular thought.
There might be different competing thought forms in play, where we'll try to identify the most prominent thought form
that is currently being expressed.
From this, a process of mind opening or mind stretching is started.
Knowing the presence of thought-forms, and in particular the absence of others, provides the opportunity to introduce
new ones by the formulation of mind-opening questions.
An example of such questions will be presented later.
