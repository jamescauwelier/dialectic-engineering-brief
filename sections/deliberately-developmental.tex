\section{Methods of Deliberate Development}

First, let's talk about the why of \emph{Dialectic Engineering}, which is appropriately described as a deliberate
developmental practice as introduced by Kegan and Lahey~\citep{kegan2016everyone}.
Often, developmental practices are mistaken for learning activities, so let's start by creating this distinction.

Laske refers to two types of developmental: vertical and horizontal~\citep{laske_MHD_vol_II}.
The first one, horizontal development, is behavioral.
This is where we learn new skills and is what most mean when discussing professional development: learning a new programming language (e.g. Go,
Typescript, Rust, etc.), a framework (e.g. Spring boot, React), a methodology (e.g. Domain Driven Design, Test Driven
Development), a paradigm (e.g. functional programming).

But true development is vertical and implies new levels of emotional and cognitive development are reached.
Learning happens when what we know changes.
True vertical development happens when how we know changes.
Using a Hegelian dialectic conversation, a topic can be detailed across 4 categories of thought-forms where the
introduction and deepening of a thought-form is an act of vertical cognitive development.
Participants need the emotional maturity to present with humility and a readiness to receive critical inquiry, after
which the rewards will be plenty.

\subsection{Thought Forms}

The thought-forms we use in our developmental model are separated according to the 4 moments of the dialectic~\citep{
    laske_2023_4_moments} where each focuses on a different category of thought:

\begin{itemize}
    \item \emph{Process}: focus on identity, change, history and evolution of thought, negation
    \item \emph{Context}: focus on parts, collections, and organizations thereof in stable constellations
    \item \emph{Relation}: focus on relating items using common ground, detailing relationships, and eventually
    seeing relationship as constitutive (being foundational of what it relates)
    \item \emph{Transformation}: focus on instabilities as the driver of transformation, systems of systems, open
    incompletely defined systems, etc.
\end{itemize}

Most focus their thoughts almost exclusively on contextual thinking.
And all of these thought-forms are abstract and sometimes elusive, but that's only because we are not accustomed to
take our own thinking as a subject of thought.
We're looking to change that to guide us to some real tangible and practical benefits.

In our sessions, we'll introduce the categories and thought-forms one by one, and practice them together.
To do this, we need another tool, the thinking spaces.

\subsection{Thinking Spaces}\label{subsec:Thinking Spaces}

A thinking space is an artificial delineation of where to apply our focus.
The distinction is artificial because our thinking doesn't follow such clear boundaries.
Focussing on a few defined thinking spaces allows us to direct attention, and protect us from engaging too much with
unproductive, philosophical outputs.

As an engineer, I get exposed to three thinking spaces:

\begin{itemize}
    \item \emph{Software (or solution domain)}: I think about how to solve challenges by writing software systems
    \item \emph{Problem Domain}: I think about the customer or users' needs and understand them deeply
    \item \emph{Organization}: I think about how the team can organize itself around problem and solution, and how
    this organization helps our hurts our progress.
\end{itemize}

It helps to know in what space we're thinking.
To practice the thought forms, we start in the solution space and make assumptions on the other spaces.
In applied practice, however, dialectic engineering will need to focus on all three dimensions and on different
levels of abstraction.